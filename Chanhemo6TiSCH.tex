% Copyright 2004 by Till Tantau <tantau@users.sourceforge.net>.
%
% In principle, this file can be redistributed and/or modified under
% the terms of the GNU Public License, version 2.
%
% However, this file is supposed to be a template to be modified
% for your own needs. For this reason, if you use this file as a
% template and not specifically distribute it as part of a another
% package/program, I grant the extra permission to freely copy and
% modify this file as you see fit and even to delete this copyright
% notice. 

\documentclass{beamer}

% There are many different themes available for Beamer. A comprehensive
% list with examples is given here:
% http://deic.uab.es/~iblanes/beamer_gallery/index_by_theme.html
% You can uncomment the themes below if you would like to use a different
% one:
%\usetheme{AnnArbor}
%\usetheme{Antibes}
%\usetheme{Bergen}
%\usetheme{Berkeley}
%\usetheme{Berlin}
%\usetheme{Boadilla}
%\usetheme{boxes}
%\usetheme{CambridgeUS}
%\usetheme{Copenhagen}
%\usetheme{Darmstadt}
%\usetheme{default}
%\usetheme{Frankfurt}
%\usetheme{Goettingen}
%\usetheme{Hannover}
%\usetheme{Ilmenau}
%\usetheme{JuanLesPins}
%\usetheme{Luebeck}
\usetheme{Madrid}
%\usetheme{Malmoe}
%\usetheme{Marburg}
%\usetheme{Montpellier}
%\usetheme{PaloAlto}
%\usetheme{Pittsburgh}
%\usetheme{Rochester}
%\usetheme{Singapore}
%\usetheme{Szeged}
%\usetheme{Warsaw}

\title{Wireless Communication and Applications}
% A subtitle is optional and this may be deleted
\subtitle{6TiSCH}

\author{Wilson Chanhemo}
% - Give the names in the same order as the appear in the paper.
% - Use the \inst{?} command only if the authors have different
%   affiliation.

\institute[University of Dodoma] % (optional, but mostly needed)

% - Use the \inst command only if there are several affiliations.
% - Keep it simple, no one is interested in your street address.

\date{Individual Assignment, October 2015}
% - Either use conference name or its abbreviation.
% - Not really informative to the audience, more for people (including
%   yourself) who are reading the slides online

% This is only inserted into the PDF information catalog. Can be left
% out. 

% If you have a file called "university-logo-filename.xxx", where xxx
% is a graphic format that can be processed by latex or pdflatex,
% resp., then you can add a logo as follows:

% \pgfdeclareimage[height=0.5cm]{university-logo}{university-logo-filename}
% \logo{\pgfuseimage{university-logo}}

% Delete this, if you do not want the table of contents to pop up at
% the beginning of each subsection:
\AtBeginSubsection[]
{
  \begin{frame}<beamer>{Outline}
    \tableofcontents[currentsection,currentsubsection]
  \end{frame}
}

% Let's get started
\begin{document}

\begin{frame}
  \titlepage
\end{frame}

\begin{frame}{Outline}
  \tableofcontents
  % You might wish to add the option [pausesections]
\end{frame}

% Section and subsections will appear in the presentation overview
% and table of contents.
\section{Introducing 6TiSCH}
\section{Why 6TiSCH}
\section{Current Proposals on Scheduling}
\section{Research Challenges/Open Areas}
\begin{frame}{Introducing 6iTSCH}
  \begin{itemize}
  \item {
   6TiSCH is IPV6 over IEEE 802.15.4e TSCH.
  }
  \item {
    The aim of 6TiSCH standard is to use features from IEEE 802.15.4e TSCH to ensure:
    high throughput, reliability and power economy}
    	\item {
    		Will enable upper layer protocols such as 6loWPAN, RPL and CoAP to work with lower layers	}
  \end{itemize}
\end{frame}

% You can reveal the parts of a slide one at a time
% with the \pause command:
\begin{frame}{Why 6TiSCH}
  \begin{itemize}
  \item {
    IEEE802.15.4e TSCH defines what a node does to execute a schedule, but does not detail how to build and maintain that schedule.
  }
  \item {   
    Similarly, an IETF standard such as RPL organizes an existing topology into a multihop routing structure, but is agnostic to the underlying link layer technology, and hence to the notion of a TSCH communication schedule.
  }
  \item{
    The two above need to be “glued” in some way.
  }
  \end{itemize}
\end{frame}
\begin{frame}{Why 6TiSCH cont...}
	\begin{figure}
\centering
\includegraphics[width=0.7\linewidth]{pic1}
\caption{Research gap}
\label{fig:pic1}
\end{figure}
\end{frame}
\begin{frame}{6TiSCH Protocol Stack[Envisioned]}
 \begin{figure}
\centering
\includegraphics[width=0.7\linewidth]{pic2}
\caption{Envisioned IETF 6TiSCH stack}
\label{fig:pic2}
\end{figure}
\end{frame}

\begin{frame}{Current Proposals on Scheduling}
\begin{itemize}
	\item{Centralized Scheduling.\\
		
			Palattella et al., Traffic Aware Scheduling Algorithm for Multi-Hop IEEE 802.15.4e Networks, Personal Indoor and Mobile Radio Communications (PIMRC), 2012 IEEE 23rd International Symposium on, 2012, pp. 327-332}
		\item {Distributed Scheduling\\
			D.Dujovne et al.,6TiSCH On-the-Fly Scheduling, Internet Draft [Work in progress], IETF Std., Rev. draft-dujovne-on-th-fly-02, 14Feb.2014\\}		
\end{itemize}
\end{frame}
\begin{frame}{Research Challenges}
	\begin{itemize}
		\item {Optimization between different protocol interactions
			\\Z. Shelby et al., Neighbor Discovery Optimization for IPv6 over Low-Power Wireless Personal Area Networks (6LoWPANs), IETF 6LoWPAN Std. RFC4861,sept 2007 }
		\item{Dynamic allocation of time slots in distributed scheduling\\
			Shanjiang Tang, Bu-Sung Lee, Bingsheng He DynamicMR: A Dynamic Slot Allocation and Scheduling Framework for MapReduce Clusters
			}
			\item{Security issues for data protection and Communication link protection}
	\end{itemize}
	
\end{frame}	
\begin{frame}{Questions}
	\begin{itemize}
		\item\centering{?}
	\end{itemize}
	
\end{frame}		
\end{document}



